%% start of file `template.tex'.
%% Copyright 2006-2013 Xavier Danaux (xdanaux@gmail.com).
%
% This work may be distributed and/or modified under the
% conditions of the LaTeX Project Public License version 1.3c,
% available at http://www.latex-project.org/lppl/.


\documentclass[11pt,a4paper,sans]{moderncv}        % possible options include font size ('10pt', '11pt' and '12pt'), paper size ('a4paper', 'letterpaper', 'a5paper', 'legalpaper', 'executivepaper' and 'landscape') and font family ('sans' and 'roman')

% moderncv themes
\moderncvstyle{banking}                            % style options are 'casual' (default), 'classic', 'oldstyle' and 'banking'
\moderncvcolor{blue}                                % color options 'blue' (default), 'orange', 'green', 'red', 'purple', 'grey' and 'black'
%\renewcommand{\familydefault}{\sfdefault}         % to set the default font; use '\sfdefault' for the default sans serif font, '\rmdefault' for the default roman one, or any tex font name
%\nopagenumbers{}                                  % uncomment to suppress automatic page numbering for CVs longer than one page

% character encoding
\usepackage[utf8]{inputenc}                       % if you are not using xelatex ou lualatex, replace by the encoding you are using
%\usepackage{CJKutf8}                              % if you need to use CJK to typeset your resume in Chinese, Japanese or Korean

% adjust the page margins
\usepackage[scale=0.75]{geometry}
%\setlength{\hintscolumnwidth}{3cm}                % if you want to change the width of the column with the dates
%\setlength{\makecvtitlenamewidth}{10cm}           % for the 'classic' style, if you want to force the width allocated to your name and avoid line breaks. be careful though, the length is normally calculated to avoid any overlap with your personal info; use this at your own typographical risks...

% personal data
\name{DEVIN}{MATTÉ}
%\title{Resumé title}                               % optional, remove / comment the line if not wanted
%\address{street and number}{postcode city}{country}% optional, remove / comment the line if not wanted; the "postcode city" and and "country" arguments can be omitted or provided empty
\phone[mobile]{+1~(203)~470~1323}                   % optional, remove / comment the line if not wanted
%\phone[fixed]{+2~(345)~678~901}                    % optional, remove / comment the line if not wanted
%\phone[fax]{+3~(456)~789~012}                      % optional, remove / comment the line if not wanted
\email{devinmatte@mail.rit.edu}                               % optional, remove / comment the line if not wanted
\homepage{devinmatte.com}                         % optional, remove / comment the line if not wanted
%\extrainfo{additional information}                 % optional, remove / comment the line if not wanted
%\photo[64pt][0.4pt]{picture}                       % optional, remove / comment the line if not wanted; '64pt' is the height the picture must be resized to, 0.4pt is the thickness of the frame around it (put it to 0pt for no frame) and 'picture' is the name of the picture file
%\quote{Some quote}                                 % optional, remove / comment the line if not wanted

% to show numerical labels in the bibliography (default is to show no labels); only useful if you make citations in your resume
%\makeatletter
%\renewcommand*{\bibliographyitemlabel}{\@biblabel{\arabic{enumiv}}}
%\makeatother
%\renewcommand*{\bibliographyitemlabel}{[\arabic{enumiv}]}% CONSIDER REPLACING THE ABOVE BY THIS

% bibliography with mutiple entries
%\usepackage{multibib}
%\newcites{book,misc}{{Books},{Others}}
%----------------------------------------------------------------------------------
%            content
%----------------------------------------------------------------------------------
\begin{document}
%-----       letter       ---------------------------------------------------------
% recipient data
\recipient{Residence Life}{}
\date{\today}
\opening{Dear Sir or Madam,}
\closing{Thank you,}
%\enclosure[Attached]{Resume}          % use an optional argument to use a string other than "Enclosure", or redefine \enclname
\makelettertitle


	I have wanted to be a RA for some time, and as a student here in the dorms for the last year, I have seen my fair share of residence life in RITs dorms. I would love an opportunity to give back in a more meaningful way to the community, and I believe becoming a RA is that opportunity.


	In my life so far I've had a series of positions that have prepared me for the role of RA. I have held a decent number of leadership positions, both in high school and since arriving here at RIT. I'm currently the financial director and insights director for Computer Science House (CSH) as well as the former secretary. Before then in high school I was in leadership positions in each of the clubs I was a part of as vice president in my technology team and secretary in the class council. I've also held jobs as a supervisor, where I was in charge of a large group of co-workers and having to manage and take care of a store. I think each of these experiences has prepared me to be an RA if not through community leadership gained from clubs, or from job responsibility which I've learned not just as a supervisor, but in all positions I've held.


	With the RA position, I would hope to learn to provide a sense of community for those I might not consider my friends. In CSH or my high school clubs, everyone that I was in a position of leadership over put me there. However as an RA, these students would look up to me, and come to me for council whether or not they would have chosen me to hold that position. I hope that the experiences that come with being an RA grant me skills in conflict resolution and with communicating with those that might not wish for me to have the position that I do.

The Center for Residence Life Core value I connect with most is certainly Community Engagement. As someone who’s lived in the dorms since arriving at RIT, I’ve seen the benefit of a floor community on all of its residents. Whether it was my original housing on 5th floor NRH or my current vibrant community in CSH, having a community where you live is incredibly important to the college experience. When you live with a large group of your closest friends and classmates, it makes coming back from class just that much more enjoyable. I definitely know that I would not have enjoyed a lot of my time at RIT so far if I was not a part of one of those communities, and I know that as a RA, I’d work hard to foster that sense of community wherever I ended up.


	I seriously hope you consider me for the position of RA. I have wanted to be able to give back to RIT’s community since I arrived here, and I hope that you’ll give me the opportunity to do so.

\makeletterclosing

\end{document}


%% end of file `template.tex'.
